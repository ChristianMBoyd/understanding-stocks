\section{Banks create money}

Mostly, I want to put the ideas and findings of Richard Werner's {\it Can banks individually create money out of nothing?} \cite{Werner2014} into my own words.  I think  Werner is intentionally provocative due to decades of frustration arguing against academic claims that lack real-world (i.e., empirical) backing.  I find a lot of Werner's arguments tautological and somewhat repetitive in nature, though I'm sure he needed to present his claims in an undeniable fashion.  Nevertheless, I also believe that the paper demonstrates how banks make loans in practice and want to flesh out my personal understanding of how {\it individual} banks create money.


\subsection{The separation of loans from deposits}

One of Werner's main points is that deposits and loans are not restricted by one another in the real banking system.  In the naive explanation of banking,\footnote{Without providing a specific source, I've heard similar explanations in modern media discussions of banking, financy, and money in general.  The goal of this example is to contrast with the ideas proposed by Werner, rather than ``disprove'' a particular banking theory or one of its proponents.} banks generate profit due to the difference in the interest rates of loans they extend and the interest they pay to their depositors.  In this model, loans are extended because there are deposits sitting around in the bank burning a hole in the bank's pocket.  Since loans in this model are backed by deposits, a deposit needs to enter the bank vaults (physically or digitally) before a loan can be extended.  Notably, this concept appears in both the {\it financial intermediary} and {\it fractional reserve} theories of banking proposed by Werner as alternatives to the credit creation theory of banking.\footnote{The details of these alternatives are thoroughly reviewed in \cite{Werner2014}.  I don't claim that banks don't operate according to a fractional reserve policy with respect to {\it deposits}, but I want to clarify that this aspect of banking is not directly related to loan--and therefore, money--creation.}  

To demonstrate the apparently-startling detachment of loans from deposits in reality, Werner applies for--and receives--a loan from a private bank in Germany.  At no point were the bank {\it deposits} checked or authorized during the loan approval account creation.  Werner claims to then go and spend this money to verify that the subsequent loan-funded account indeed contained real money that could be used in the economy.  Naturally, if the naive explanation of banking held true, then a bank would either have deposits lying around that covered the new loan or it would need to drum up new deposits commensurate with the loan requested.  In both cases, the bank's deposits would need to be checked before the loan could be extended to Werner.


\subsubsection{Bank loan accounting}

The fractional reserve and financial intermediary theories of banking--as covered by Werner--emphasize the point that the money funding a loan must come from {\it somewhere}.  Therefore, it's natural to suspect that this money comes from deposits if banks are primarily in the business of holding deposits.  On the contrary, Werner has repeatedly stated in interviews \note{[add at least one example]} that banks are not in the deposit-lending business, but the {\it securities} business.  By removing the assumed role played by deposits in the lending business, we can more critically analyze the value exchanged when a bank makes a loan.  

Banks create loans when the bank and a borrower come to an agreement based on the borrower's credit worthiness and ability to generate sufficient (future) funds to repay the loan.  For lending to be a profitable business, banks must ensure that the loan terms they offer (maturity, payment schedule, interest, etc.) are sufficient cover their risk (borrowers defaulting) with enough margin built in so that they receive sufficiently more than they lend on average.\footnote{I.e., banks must cover their own expenses associated with loan creation to remain in business.}  When a loan is trusted to be profitable, the loan {\it itself} (i.e., the repayment contract between the borrower and the bank) provides the value necessary to fund it.  This line of reasoning underlies the concept of {\it money creation} by individual, private banks.  

As Werner demonstrates the mechanics of bank loans \cite{Werner2014}, he finds that--to a reasonable level of certainty--the bank counts the loan as an asset that offsets the newly-created deposit associated with his (now funded) bank account.  By contrast, the deposit is counted as a liability on the bank's balance sheet.\footnote{There are general remarks in \cite{Werner2014} related to how banks generally count deposits as liabilities (i.e., loans made {\it to} the bank that must be repaid), but for now I'm only focusing on the deposit associated directly with the loan funds.}  Of course, when the funds associated with the newly-created deposit are withdrawn, that is an obligation that must be paid by the bank to another institution (likely another bank).  {\it This} outbound money must come from somewhere.  The point made by Werner \cite{Werner2014} is that this short-term liability is simply part of the bank's broader accounts payable.  His takeaway is that funding bank loans is a part of the bank's day-to-day business expenses, which is independent of its deposits or--more generally--it's reserve requirements as a deposit-holding institution.\footnote{By ``reserves'' here, I mean the assets (liquid or otherwise) used to prove that the bank is not insolvent.  Since existing loans are typically already extended (i.e., funded), there is presumably no direct connection between existing loans and the concept of ``reserves.''}


\subsection{The self-sustaining business of bank loans}

\note{[Todo]}:
\begin{itemize}
    \item Flesh out how a bank with sufficient, profitable loans (already extended) funds new loans in a completely self-containg, self-propagating cycle.
    \item Make a separate section (later, or here) on how a bank goes from a deposit-holding institution with 0 loans to a fully self-sustaining loan machine.  I.e., explain how banks bootstrap themselves into a money-creating instutition.
\end{itemize}


\subsubsection{A small-scale analysis: when does the loan business pay off?}

This is a relatively loose, back-of-the-napkin calculation I want to have on-hand.  I'd consider the loan business worthwile if I could pocket \$100,000/year.  A Google search suggests that small business loans tend to carry interest rates around 6\%-11\% from a typical bank.  To establish a conservative estimate, let's say the average return on already-extended loans is 5\%.  Generating \$100,000 per year {\it in profit} then requires \$2,000,000 in extended loans if the principal didn't decrease.  Another Google search suggests that typical business loans are between 3-10 years.  To keep it simple, let's say the average principal drain on already-extended loans is roughly captured by the repayment schedule 2.5 years into a 5 year term on a {\it single} \$2,000,000 loan.  Using an online amortization calculator, this looks like a principal drain of nearly \$400,000 per year, or 20\% of the total amount of extended loans.

\note{[Todo]}:
\begin{itemize}
    \item Where do the new loans come from?
    \item Does this example scale with extended loans, or are the interest rates or amortization themselves over-simplified to the point where it's not consistent?
\end{itemize}


\subsection{Todo/further notes}

\note{[Todo]}:
\begin{itemize}
    \item Find at least one reference where Werner mentions that banks are in the securities business to add some validity to this argument on his behalf.
\end{itemize}