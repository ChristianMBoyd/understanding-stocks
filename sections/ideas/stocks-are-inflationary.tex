\subsection{Stocks are inflationary}

An idea I can't shake is that I would expect the main tenets of stock investment\footnote{Here, I mean investment rather than speculation, as outlined in \cite{Graham2024}.} to be strictly inflationary to the overall economy.  In particular, I'm confused by the idea that long-term investment into a stock should reward the investor in stock price rather than dividend payments, or a bond-like payout schedule.  I don't doubt that this strategy works empirically, or that it's mutually beneficial between the shareholders and the company issuing stock.  Rather, I fail to see the societal or ``real economy'' benefit from stock investing based on owning future profits.

\subsubsection{The company-shareholder dynamic}

A quote from Warren Buffet roughly states that {\it ``In the companies we buy, the management team goes to work because they want to, not because they have to.  The [original] owners work weekends and evenings out of their own passion''} \note{[Find this video and link]}.  This quote has stayed with me ever since I heard it, because it highlights that {\it management}\footnote{By {\it management}, I mean whoever runs the company and makes business decisions that impact customers.  This may include owners, executives, or high-achieving employees with a strong customer-facing presence.} is who pours their blood, sweat, and tears into creating real value that benefits their customers; shareholders have merely purchase the right to the profits of their labor.  I doubt highly that Warren Buffet is putting his weekends and evenings into whatever company was being discussed in the quote, but he nevertheless stands to reap the benefits.

The concept of buying future profits bothers me---not because I'm a communist, but because of how value is derived and who holds the short end of the stick if things go astray.  To make my concerns concrete, I want to consider two different scenarios: one in which I buy a bond issued by a company and one in which I buy company stock.  I want to dive into the upside and downside of each interaction, for both the investor and the company.  For now, I am considering the case where I purchase a bond issued directly from the company, rather than from another investor.

If I buy a bond issued by a company, then I receive a well-defined return (interest) based on the financial credibility of the company and the duration of the bond maturity.\footnote{For simplicity, but additional details such as these would impact the precise valuation of the bond issued, rather than the nature of the subsequent argument.}  In the case that the company fails, I--the investor--hold the short end of the stick and must eat the losses associated with the remaining obligation on the loan or bond.  In the case that all goes well, then I receive the agreed-upon benefit at term maturity.  Notably, a company only issues bonds if the current value of that cash injection will (likely) generate profits that exceed its maturity obligations.  Necessarily, then, that company must be doing {\it something} with that capital that generates customer value in excess of the bond investment.  Even if the company is in financial trouble, the ability for immediate cash injection to outweigh future obligations is a necessary calculation for both the company and investor to understand before agreeing to the bond terms.  As a result, I see bond issuance as a purely productive investment that necessarily creates more value than it costs if we assume that bonds reach maturity sufficiently often to justify their valuations.  

If I buy a company, or purchase sufficient shares to become an influential shareholder, then I am buying both the current assets and the future profits of that company.  For example, if the future profit forecast takes a downturn in one year's time, then I--the whole or part owner--am presumably underwater on the asset I bought.  The difference, however, is that I don't just own a piece of paper demanding the company pay me a fixed obligation: I own the company itself and can direct management to make changes that better align with my profit-seeking expectations.  Notably, this circumstance can befall an investor (or owner, etc.) even if all business obligations are met and profits are stable or even {\it increasing}; what's gone ``wrong'' is that the generated profits are not to the level of my and my fellow shareholders' liking.\footnote{Alternatively, the generated profits are not to the level where other market participants would be willing to pay a premium for stock in the company.}  The result of my stock investment is to generate an invisible liability on the company where, if profits do not exceed metrics associated with the current stock price, I will begin meddling with company affairs.  Of course, the stock may go to zero and nobody will weep for the investor left holding the paper company; however, my concern of inflationary behavior corresponds to this invisible threat of shareholder intervention should profits not meet their standards.

Whereas company bond issuance is necessarily preceded by value-generating opportunity, stock investment demands after-the-fact profit seeking despite no actual investment into the company's profit-generating capacity.  Primary bond investment is a transaction where investors transfer present-day cash into a company in exchange for future cash.  Stock investment is a transaction between current and future owners.

\note{[Todo]:}
\begin{itemize}
    \item \note{[Dividends and buybacks]:} discuss dividends and stock buybacks in the context of returning value to shareholders.  Do these help the situation or worsen the inflationary argument?
    \item \note{[Inflation]:} clearly define inflation and why something might be ``inflationary.''  Am I considering something more specific like cost of living vs. income?  Am I considering the U.S.'s inflation index?
    \item \note{[IPOs]:} give a generous discussion of why companies go public at all.  There must be some benefit -- or do I really believe that it's simply a means of owners and early investors cashing out?  Look into the origins of public stock issuance; have things changed since then?
    \item \note{[Company liquidity]:} what are my thoughts on the ability for entrepreneurs to sell their ownership stake?
    \item \note{[Returning shareholder value]:} do non-inflationary mechanisms exist to return value to shareholders?  How can shareholders extract company profits without damaging the business?  \note{[Can ignore if this is answered by dividends/buybacks.]}
\end{itemize}