\section{Reading notes}

This section contains specific references I want to further discuss, or simply notes on references I want to write down.


\subsection{Money creation in the modern economy (M. McLeay, A. Radia, \& R. Thomas)}

After reading \cite{BOE2014}, I felt that there wasn't much substance to the actual article; however, it has led to some other references on money supply and inflation.


\subsection{Will the securitization revolution spread? (R. Feldman)}

In \cite{Feldman1995}, Feldman considers why small business loans have not been securitized to the extent of mortgages, auto loans, and credit card receivables.  This discussion, however, is from 1995 and so this should be more of a jumping-off point than taken as an explanation of the modern situation for small business loan securitization.

Feldman notes that the benefits to securitization are obvious (increased liquidity for lenders, increased loans for borrowers, stable income opportunities for investors), so a natural suspect for the lack of adoption is the difficulty (i.e., cost) in securitization.
\begin{itemize}
    \item Small business loans require detailed information on the management team and the local economy in which the business operates.
    \item Small business loans lack (or, at least they might've in the past) standardized terms.
    \item There is a lack of publicly-available, long-term performance data on small business loans.  I.e., it's a more opaque market when compared to mortgage loans.
\end{itemize}

I find this first point interesting, as it should presumably apply for mortgages and credit card receivables.  I.e., a bank requires detailed income and credit history for a mortgage applicant and a credit card company needs the same to establish a revolving line of credit.  \note{[Consider]:} {\it how does the future income calculation differ between a small business and an individual?}  Feldman notes that this cost/lack of understanding is mostly on the investor side (lenders are already capable of making the loans), who will choose a better-understood investment like mortgage-backed securities when given the option.

Feldman further notes that investors want loss protections when they don't understand the mechanics of the securities they're purchasing.  These loss protections, however, imply that the bank (or, more generally, the lender) still carries the credit risk of the securitized loans.  Maintainin the credit risk requires banks to buy costly insurance against losses on securitized loans.  \note{[Consider]:} {\it how do SBA 7(a) guarantees help with this?}  Moreover, Feldman states that bank regulations consider loans that still carry credit risk to have not been fully sold; i.e., the bank must still meet capital requirements associated with having the securitized loans on its books.  From the bank's point of view, securitization provides liquidity but does not remove the accounting complexity or credit risk associated with the loans.  From the investors point of view, small business loan securities are a complicated--and risky--asset that are not worth the trouble once all the required protections are priced into the return.  It's re-emphasized throughout that the informational burden on the investor is the primary cost preventing broader adoption of securitized small business loans.

\note{[Competition]:} Feldman notes that small business loans are less subject to non-local (lending) competition.  Worth looking into whether this is still the case 30 years later.

\note{[Loan standards]:} it's noted that not all loans are made through the SBA program.  It's worth looking into the size of this non-SBA loan market.


\subsubsection{Summary}

This has been a thought-provoking article that fleshes out the pros and cons associated with small business loan securitization.  Feldman's points here should be contrasted with the current small business loan market.


\subsection{R. Shiller - From efficient markets theory to behavioral finance}

In \cite{Shiller2003}, Shiller puts forth that stock price volatility is at odds with the efficient market hypothesis, which he claims requires stock prices to equal the optimal forecast of the present value of all future dividends from that stock.  In particular, Shiller requires market movements in an ``efficient" market to result from new information that causes market participants to rebalance their portfolios.  Consistent volatility that exceeds informational shortcomings, however, would discredit Shiller's construction of the efficient market hypothesis.

Shiller goes on to compare plots of present value calculations against stock price trends over the past century.  He claims that there is too much volatility in the S\&P to positively support the efficient market hypothesis.  Confusingly, it seems that there is some predictive power in the price of individual stocks about their present value, but that no simple weighting or time-discounting scheme appears to reproduce stock prices.  In particular, no simple scheme of determining the present value of stock divident payouts seems to accurately track boom and bust cycles.

Instead, Shiller transitions to discussing behavioral psychology, where people are shown to exhibit irrational strategies (or heuristics) that would lend themselves to boom-bust cycles in trading markets.  He relates the fervor of a Ponzi scheme to market bubbles, where initial investors get such high returns that new investors are no longer considered with the fundamentals that created the returns.  People simply want in on the action and embrace the risk.

Shiller goes on to discuss the efficient market answer to irrational market participants: that ``smart money'' will move opposite the irrational participants to profit off their mistakes.  An issue Shiller brings up is an apparently-widespread issue that it can be difficult to exert downward pressure on an asset bubble.  He notes the example of 3Com's Palm offering, where Palm shares rose to such heights that the remaining 3Com business had an implied negative value on their overall holdings.  An issue in this bubble was that investors who realized the over-pricing of Palm could not easily short Palm stock due to shortages and high price barriers to put options.  While it's stated that this state of affairs is relatively rare, this sounds a lot like discussion around the housing bubble of the mid-00s.  Even though people had concerns about housing prices, there was no obvious/efficient mechanism for skeptical investors to push prices down.\footnote{Of course, there is the story of {\it The Big Short}, where clever investors made money through credit default swaps with unsuspecting banks.  These investors, however, did not have any actual capacity to sell the underlying homes (and push prices down).}

This paper wraps up with a summary that the stock market certainly reflects information about the companies being bought and sold.  It may be surprising in how efficient it ``normally'' operates, but it is certainly not perfectly efficient.  Shiller advises that economists (or investors, or anyone really) should keep in mind that the market prices are not perfect and that irrational behaviors are a necessary consequence of humans being the buyers and sellers in the market.  I think this paper was most interesting to me as an introduction to how Shiller--and others--value stocks, and think of them as dividend return vehicles.  Any alternative pricing strategy appears to be implicitly ``irrational.''


\note{[Consider]}:
\begin{itemize}
    \item Do I agree that the error is necessarily uncorrelated with the current price/information?
    \item E.g., if a crucial piece of information is missing from the pubilc knowledge, and that piece of information would drastically alter the price, wouldn't the error necessarily depend on the price?
    \item Cast another way: if the error is the actual minus the forecast, then how can this residual not depend on the forecast?
    \item What {\it causes} the uncertainty in $P_t^*$?  Is it purely the correct discount rate for each future dividend?  Presumably it is based off the actual dividends that are eventuallly paid out, rather than being another estimate.
    \item In what universe is the uncertainty in the present value of all future {\it known} dividend payouts bounded (below) by the uncertainty in the market forecast of that present value?  If the fed guaranteed 100 years of interest rates in quarterly intervals, how would the uncertainty in the forecasted present value of future possible dividend payments continue to bound the uncertainty in the present value of actual (known) future dividend payments?
    \item Am I misunderstanding the concept of {\it variance} in this discussion?  Why would the forecast of a random variable ever have a smaller variance (i.e., uncertainty) than the variable itself?
    \item How can it be that per-stock analyses show reasonable price forecasting through the present value, but that this apparently fails on the broader market?  Is this a tail effect where the ``market average'' is skewed by the exceptionally unpredictable stocks?  Do portfolios of stocks somehow create more uncertain forecasts than individual ones?
    \item If boom and bust cycles are removed ({\it if} they can be removed), does the stock price more accuratly reflect present value?  Are boom/bust cycles an aggregate market phenomenon that detracts from an otherwise present-value forecasted stock pricing market?
\end{itemize}


\subsection{K. Cole, J. Helwege, and D. Laster - Stock market valuation indicators: is this time different?}

This paper discusses the state of the stock market following the summer of 1995.  It's noted that the 2.5\% dividend yield in the summer of 1995 was a historical low.  It's further noted that the market-to-book ratio over 3 exceeds the historical average of 1.9.  It starts out by noting that low dividend yields and high market-to-book ratios (stock market capitalization relative to the accounting book value of a company) tend to signal poor future performance.  Counter-arguments to this thinking is allegedly that the era of tax-advantaged stock buybacks and new accounting rules have resulted in an artificial increase in market-to-book ratios and an artificial decrease in dividend yields.  {\it Note: I can't help but notice the echoes of Robert Shiller's critique of the efficient market hypothesis, where he mentioned that new techniques and rationales emerge in a bubble to support ``the new norm'' of eternally-high stock prices and metrics.}  The trio aim to rebuke these claims and sound the alarm on over-priced stocks in this paper.

{\it Continue from Share Repurchases}\ldots

\note{[Todo]}:
\begin{itemize}
    \item The S\&P {\bf 500} (note, not the S\&P 400 used for market cap analysis in the text) dividend yield was a measly 1.17\% in September 2025, an apparent historical low.  I'm guessing from the data that the norm is closer to 2\%, which is still low in the eyes of the text.  A Google search suggests a market-to-book ratio somewhere between 2.5 and 2.9, which is on the higher side but not as inflated as the text describes.  
\end{itemize}


\subsection{K. Case and R. Shiller - Is there a bubble in the housing market?}

\note{[Todo]}: Read \cite{Case2003}