\section{Reading notes}

This section contains specific references I want to further discuss, or simply notes on references I want to write down.


\subsection{Money creation in the modern economy (M. McLeay, A. Radia, \& R. Thomas)}

After reading \cite{BOE2014}, I felt that there wasn't much substance to the actual article; however, it has led to some other references on money supply and inflation.


\subsection{Will the securitization revolution spread? (R. Feldman)}

In \cite{Feldman1995}, Feldman considers why small business loans have not been securitized to the extent of mortgages, auto loans, and credit card receivables.  This discussion, however, is from 1995 and so this should be more of a jumping-off point than taken as an explanation of the modern situation for small business loan securitization.

Feldman notes that the benefits to securitization are obvious (increased liquidity for lenders, increased loans for borrowers, stable income opportunities for investors), so a natural suspect for the lack of adoption is the difficulty (i.e., cost) in securitization.
\begin{itemize}
    \item Small business loans require detailed information on the management team and the local economy in which the business operates.
    \item Small business loans lack (or, at least they might've in the past) standardized terms.
    \item There is a lack of publicly-available, long-term performance data on small business loans.  I.e., it's a more opaque market when compared to mortgage loans.
\end{itemize}

I find this first point interesting, as it should presumably apply for mortgages and credit card receivables.  I.e., a bank requires detailed income and credit history for a mortgage applicant and a credit card company needs the same to establish a revolving line of credit.  \note{[Consider]:} {\it how does the future income calculation differ between a small business and an individual?}  Feldman notes that this cost/lack of understanding is mostly on the investor side (lenders are already capable of making the loans), who will choose a better-understood investment like mortgage-backed securities when given the option.

Feldman further notes that investors want loss protections when they don't understand the mechanics of the securities they're purchasing.  These loss protections, however, imply that the bank (or, more generally, the lender) still carries the credit risk of the securitized loans.  Maintainin the credit risk requires banks to buy costly insurance against losses on securitized loans.  \note{[Consider]:} {\it how do SBA 7(a) guarantees help with this?}  Moreover, Feldman states that bank regulations consider loans that still carry credit risk to have not been fully sold; i.e., the bank must still meet capital requirements associated with having the securitized loans on its books.  From the bank's point of view, securitization provides liquidity but does not remove the accounting complexity or credit risk associated with the loans.  From the investors point of view, small business loan securities are a complicated--and risky--asset that are not worth the trouble once all the required protections are priced into the return.  It's re-emphasized throughout that the informational burden on the investor is the primary cost preventing broader adoption of securitized small business loans.

\note{[Competition]:} Feldman notes that small business loans are less subject to non-local (lending) competition.  Worth looking into whether this is still the case 30 years later.

\note{[Loan standards]:} it's noted that not all loans are made through the SBA program.  It's worth looking into the size of this non-SBA loan market.


\subsubsection{Summary}

This has been a thought-provoking article that fleshes out the pros and cons associated with small business loan securitization.  Feldman's points here should be contrasted with the current small business loan market.