\section{Securitization}

As I've read more about banking and finance, I'm drawn to the idea of securitization.  In particular, securitization sounds like a means to rapidly extend capital without needing to wait for lengthy repayment schedules.  If there is investment worth making, then the value of the loans exceeds their cost, and that value can be repackaged for investors.  So long as there is a sufficient premium available to securitizers, then securitization can efficiently distribute capital into worthwhile investments.


\subsection{Bond securitization arbitrage}

An idea that comes to mind, as a means of boot-strapping a non-bank financial company, is whether there exist straightforward securitization strategies on existing bonds.  If the bonds are already publicly traded and understood, then perhaps a securitized portfolio could be more easily accepted/bought by investors.  In particular, it may be possible to secure insurance against default on the package of publicly rated/understood bonds.  This could be a stepping stone into securitization\ldots

\note{[Todo]}: flesh this out more and try to see if it's done/exists.


\subsection{Todo/further notes}

\note{[Todo]}:
\begin{itemize}
    \item Work through an actual securitization process, noting how the borrower rate (the rate paid to the original lending institution) compares to the investor rate offered in the securities.
    \item Compare more accurate calculations to the napkin math in \ref{subsubsection: How fast can a bank grow by extending loans?}
\end{itemize}