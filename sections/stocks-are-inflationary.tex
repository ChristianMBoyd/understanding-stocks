\section{Stocks are inflationary}

An idea I can't shake is that I would expect the main tenets of stock investment\footnote{Here, I mean {\it investment} as defined by Graham \cite{Graham2024}.} to be strictly inflationary to the overall economy.  In particular, I'm perplexed by the idea that long-term investment into a stock should reward the investor in stock price rather than dividend payments, or a bond-like payout schedule.  I don't doubt that this strategy works empirically, or that it might be mutually beneficial between the shareholders and the company issuing stock.\footnote{The accuracy of this statement likely depends on how {\it the company} is defined.}  Rather, I fail to see the societal or ``real economy'' benefit from stock investing based on owning future profits.

\subsection{Stock investment vs. bond investment}

A quote from Warren Buffet roughly states that {\it ``In the companies we buy, the management team goes to work because they want to, not because they have to.  The [original] owners work weekends and evenings out of their own passion''} \note{[Find this video and link]}.  This quote has stayed with me ever since I heard it, because it highlights that {\it management}\footnote{By {\it management}, I mean whoever makes the day-to-day business decisions that impact customers.  This may include owners, executives, or high-achieving employees with a strong customer-facing presence.} is who pours their blood, sweat, and tears into creating real value that benefits their customers; shareholders have merely purchase the right to the profits of their labor.  I doubt highly that Warren Buffet invests his weekends and evenings into whichever company was being discussed in the quote, but he nevertheless stands to reap the benefits.

The concept of buying future profits irks me due to the incentive structure it presents.  To make things concrete, I want to contrast future profit ownership (stock investment) with loan extension (bond investment).  I want to dive into the upside and downside of each interaction, for both the investor and the company.  For now, I am considering the case where I purchase a bond issued directly from the company, rather than from another investor.

If I buy a bond issued by a company, then I receive a well-defined return (interest) based on the financial credibility of the company, interest rates, and the duration of the bond maturity.  In the case that the company fails, I--the investor--hold the short end of the stick and can only hope to claw back my losses from the company's assets in bankruptcy.  In the case that all goes well, then I receive the agreed-upon benefit at term maturity.  Presumably, a company would only issue bonds if the current value of that cash injection would (likely) generate profits that exceed its maturity obligations.  Necessarily, then, that company must be doing {\it something} with that capital that generates customer value in excess of the bond investment.  Even if the company is in financial trouble, the ability for immediate cash injection to outweigh future obligations is a necessary calculation for both the company and investor to understand before agreeing to the bond terms.  As a result, I see bond issuance as a purely productive investment that necessarily creates more value than it costs if we assume that bonds reach maturity sufficiently often to justify their valuations.  

If I buy a company, or purchase sufficient shares to become an influential shareholder, then I am buying both the current assets and the future profits of that company.  If the future profit forecast takes a downturn, then I--the whole or part owner--am presumably underwater on the asset I bought.  Notably, a stock investor (or owner, etc.) could end up in this circumstance even if all business obligations are met and profits are stable or even {\it increasing}; what's gone ``wrong'' is that the generated profits are not to the level of my and my fellow shareholders' liking.\footnote{More accurately, the generated profits are not to the level where other market participants would be willing to pay a premium for stock in the company.}  Because I own the company, I need not be a bystander to my dwindling asset and can direct management to increase their profit-seeking activities until my appetite for future profits is satiated.  If the future profit forecast increases, then I have gained paper value in the sense that reasonable market participants should be willing to pay more for my stock than I spent on it.  Because the payoff in stock investment results from profit generation, rather than producing it, I suspect stock investment is purely inflationary.

At least naively, the cause-and-effect of bond vs. stock investment appears flipped.  
Whereas company bond issuance is necessarily preceded by value-generating opportunity, stock investment demands after-the-fact profit seeking despite no actual investment into the company's profit-generating capacity.  Primary bond investment is a transaction where investors transfer present-day cash into a company in exchange for future cash.  Stock investment generates an invisible liability on the company where, if profits do not exceed metrics associated with the current stock price, owners are incentivized to begin meddling with company affairs.

\subsubsection{An ``investment'' tale about Benjamin Graham}

In Roger Lowenstein's {\it Introduction to Part I} of {\it Security Analysis} \cite{Graham2023}, Benjamin Graham's lifetime of investment is recounted.  A story that I feel highlights my concerns about the Graham style of ``investing'' involves a 1926 foray into Nothern Pipeline, a spin-off of Standard Oil.  Graham discovered that Nothern Pipeline's financial portfolio included securities that, on their own, significantly exceeded the company's current valuation.  Graham's clever ``investing'' involved purchasing a sufficient number of shares to pressure the company into selling off their securities and returning a windfall dividend to its ``investors.''  When management at Northern Pipeline resisted Graham's self-interested suggestions, he fought for--and won--a board seat, after which he subsequently enacted his self-enriching plan.

Graham's raiding of Northern Pipeline's coffers demonstrates the mismatched incentives of ``value'' ````investing.''''  While nobody should weep for the last tendrils of Standard Oil, the principles applied to Graham's corporate plundering were not due to the company's dubious morality or its monopolistic upbringing.  The belief of Graham, and the other investors who eventually supported him, was that Northern Pipeline's use of its own financial holdings was inferior to the investor's use of that capital at earning them a return.  In other words, Northern Pipeline didn't need its money for day-to-day operations, and so it should fork it over to the investors.  This is a conclusion that no management team would ever arrive at, and yet it seems perfectly reasonable from the perspective of ``investors.''  In order to earn a profit, investors demand the company they've bought to return more value than its price.


\subsection{Todo/further notes}

\note{[Todo]:}
\begin{itemize}
    \item \note{[Dividends and buybacks]:} discuss dividends and stock buybacks in the context of returning value to shareholders.  Do these help the situation or worsen the inflationary argument?
    \item \note{[Inflation]:} clearly define inflation and why something might be ``inflationary.''  Am I considering something more specific like cost of living vs. income?  Am I considering the U.S.'s inflation index?
    \item \note{[IPOs]:} give a generous discussion of why companies go public at all.  There must be some benefit -- or do I really believe that it's simply a means of owners and early investors cashing out?  Look into the origins of public stock issuance; have things changed since then?
    \item \note{[Company liquidity]:} what are my thoughts on the ability for entrepreneurs to sell their ownership stake?
    \item \note{[Returning shareholder value]:} do non-inflationary mechanisms exist to return value to shareholders?  How can shareholders extract company profits without damaging the business?  \note{[Can ignore if this is answered by dividends/buybacks.]}
\end{itemize}